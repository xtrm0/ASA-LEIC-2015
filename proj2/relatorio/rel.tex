\documentclass{scrartcl}
\usepackage{fontspec} %connects to native fonts
\usepackage{amsmath}
\usepackage{mathtools}
\usepackage{cleveref}
\usepackage{pgfplots}
\usepackage{graphicx}
\usepackage{wrapfig}
\usepackage{fancyref}
\usepackage{amssymb}
\usepackage{subfig}
\usepackage{float}
\usepackage[justification=RaggedRight, singlelinecheck=false, font={footnotesize}]{caption}
\usepackage[portuguese]{babel}
\usepackage[title,titletoc,toc]{appendix}


\usepackage{lipsum}
\usepackage{blindtext}
\addtokomafont{sectioning}{\rmfamily}

\begin{document}
\pagenumbering{arabic}
\bibliographystyle{plain}
\title{
	\textnormal{
	\LARGE Universidade de Lisboa - Instituto Superior Técnico\\
	\Large Licenciatura em Engenharia Informática e de Computadores\\
	\Large Análise e Síntese de Algoritmos
\\}
	\LARGE2º Projeto
	\vspace{-1ex}
	}
\author{Nuno Amaro,
	\texttt{81824}
	\and
	Afonso Tinoco,
	\texttt{81861}
}
\date{	\vspace{-1ex}
		\vspace{-4ex}
	}
\maketitle
%		{\large Universidade de Lisboa}\\[0.4cm]
%		{\large Instituto Superior Técnico}\\[0.4cm]
%		{\large Licenciatura em Engenharia Informática e de Computadores}\\[1.5cm]
\section*{Introdução}
Com este projeto pretendemos expor um algoritmo em tempo linear para o problema proposto, explicar a sua implementação e fazer uma análise teórica e experimental da complexidade temporal e espacial deste.

\section*{Descrição do problema}

\section*{Algoritmo utilizado}

\section*{Estruturas utilizadas:}

\section*{Explicação do algoritmo}

\section*{Prova de correção do algoritmo}

\section*{Análise assintótica temporal téorica do algoritmo}

\section*{Análise assintótica temporal experimental do algoritmo}

\section*{Análise assintótica espacial}

\section*{Prova de otimalidade do algoritmo}

\bibliography{ref}

\end{document}
